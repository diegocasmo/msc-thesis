\section{The \mlblink Algorithm} \label{sect:theory:ml-blink}

% Let $\vect{x}$ and $\vect{y}$ represent two images of the same location of the night sky from distinct datasets encoded as vectors. The goal is then to create a scoring function which is able to recommend objects that are more likely to contain an anomaly than those that are not. Let the scoring function be defined as in equation ~\ref{eq:v-explicit}, where $\vect{D}$ is the matrix that needs to be learned.

% \begin{equation} \label{eq:v-explicit}
%   v = \vect{x}^T \vect{D} \vect{y}  
% \end{equation}

\todo[inline]{TODO: Define vectors $\vect{x}$ and $\vect{y}$}
\todo[inline]{TODO: Define weights matrix $\vect{D}$}
\todo[inline]{TODO: ``Flesh--out'' $v = \vect{x}^T \vect{D} \vect{y}$}
\todo[inline]{TODO: Explain intuitively what does each element of $\vect{D}$ mean}
\todo[inline]{TODO: Explain how the \mlblink algorithm uses semi--supervised learning and active learning}
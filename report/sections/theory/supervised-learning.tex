\section{Supervised Learning} \label{sect:theory:supervised-learning}

Supervised learning is a function--fitting paradigm, where a model of the form $\vect{Y} = f(\text{X}) + \epsilon$ is a fair premise. The goal of supervised learning is to learn $f$ through a ``teacher'', which usually consists of a set of training observations of the form $\uptau = (x_i, y_i), i = 1,..., N$ where $x_i$ is an input pattern and $y_i$ is its corresponding label \cite{book:esl}. The model must also have the property that it can modify its input/output relationships in response to the differences between the predicted label and the true label of an observation. Once the learning process is completed, the expectation is that the outputs predicted by the learner will be similar to the true outputs such that the model is useful for all sets of inputs likely to be seen in practice \cite{book:esl}.
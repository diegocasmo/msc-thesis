\subsection{\mlblinkapi} \label{subsect:case-study:impl:ml-blink-api}

The \mlblinkapi implements the \mlblink algorithm as described in section \ref{ch:meth} using \python. It uses the MongoDB database in order to persist data, and the Celery asynchronous task queue to schedule and execute time consuming operations. The \mlblink algorithm implementation can be split into three parts: the active set (\ref{subsubsect:case-study:impl:active-set}), potential anomalies (\ref{subsubsect:case-study:impl:potential-anomalies}), and crawling candidates (\ref{subsubsect:case-study:impl:crawling-candidates}).

\subsubsection{The Active Set} \label{subsubsect:case-study:impl:active-set}
The active set is a collection persisted in MongoDB compromised of missions which are considered to be non-anomalies. Table \ref{table:case-study:impl:active-set:schema} shows the schema definition of a document of the active set.

\begin{table}[H]
    \centering
        \begin{tabular}{| l | l |} 
            \hline
                \texttt{image\_key} & An integer which identifies an image \\
            \hline
                \texttt{usno\_band} & The \usno band of the image \\
            \hline
                \texttt{panstarr\_band} & The \panstarrs band of the image \\
            \hline
                \texttt{usno\_vector}  & \multicolumn{1}{m{8cm}|}{The resulting vector after processing the original \usno image.} \\
            \hline
                \texttt{panstarr\_vector} & \multicolumn{1}{m{8cm}|}{The resulting vector after processing the original \panstarrs image.} \\
            \hline
        \end{tabular}
    \caption{A description of the schema of a document of the active set collection.}
    \label{table:case-study:impl:active-set:schema}
\end{table}

When a user submits a mission in the \mlblinkui, its data is received by the \mlblinkapi and persisted in the database. Following that, a background task is created to further determine what to do with it. \newline

The background task, named \texttt{tprocess\_created\_mission}, is in charge of defining whether a mission's data should be added to the active set or not. A mission is added to the active set if its~\texttt{accuracy} achieved in the \mlblinkui is at least as good as its~\texttt{accuracy\_threshold}. Additionally, the mission's data is analyzed to determine whether the user tried to at least do a matching by verifying whether the two images coordinates overlap (i.e. one image is placed on top of the other). Finally, if a mission's data is successfully inserted in the active set collection, the \texttt{tcrawl\_candidates} task is called to crawl a new candidate mission. 

\subsubsection{Potential Anomalies} \label{subsubsect:case-study:impl:potential-anomalies}

The \texttt{tprocess\_created\_mission} is also in charge of inserting missions in the potential anomalies collection. A mission is added to the potential anomalies collection when a mission's~\texttt{accuracy} is less than its~\texttt{accuracy\_threshold} and the mission's images overlap. If a mission is classified as a potential anomaly, the \texttt{tcrawl\_candidates} task is not called.

The schema definition of a potential anomaly is shown in table~\ref{table:case-study:impl:potential-anomalies:schema}. Its definition is similar to that of a document of the active set, but it differs in that it does not define the pre--processed vectors of the \usno and \panstarrs image specified by the \texttt{image\_key}, \texttt{usno\_band}, and \texttt{panstarr\_band} attributes.

\begin{table}[H]
    \centering
        \begin{tabular}{| l | l |} 
            \hline
                \texttt{image\_key} & An integer which identifies an image \\
            \hline
                \texttt{usno\_band} & The \usno band of the image \\
            \hline
                \texttt{panstarr\_band} & The \panstarrs band of the image \\
            \hline
        \end{tabular}
    \caption{A description of the schema of a document of the potential anomalies collection.}
    \label{table:case-study:impl:potential-anomalies:schema}
\end{table}

\subsubsection{Crawling Candidates} \label{subsubsect:case-study:impl:crawling-candidates}

The \texttt{tcrawl\_candidates} background task uses the Celery ``chord'' API to crawl for candidates. The Celery chord API allows to execute a task once a group of other parallel tasks have finished. \newline

The \texttt{tcrawl\_candidates} generates a list of image keys to analyze over all five bands defined among the two datasets. This list is refereed to as the potential candidates list, since it contains the candidate that will be selected when the algorithm finishes. Missions which are known to be potential anomalies are filtered out of this list, since these have already been tagged for further analysis. \newline

Once the potential candidates list has been generated and filtered, it is split into the available number of processes the machine where the \mlblinkapi is running has. Each chunk is then processed in parallel by a different Celery worker. These workers compute the value of \texttt{v} for each potential candidate in its chunk using the implicit definition described in section \ref{sect:meth:implicit}. For each potential candidate, its $v$ value is computed after it has been binarized with a fixed threshold value of $60$ for \usno and $220$ for \panstarrs. The dimensionality of these potential candidates is also reduced and finally the remaining features normalized as described in the pseudo--code in  algorithm \ref{pscode:retrieve-vector}. \newline

Finally, once all chunks have been processed, a task is used to ``reduce'' the result, by retrieving the potential candidate that has the lowest \texttt{v} value among all the potential candidates analyzed by the different Celery workers. Since it is the most different to those in the active set (and therefore most likely to contain an anomaly), this candidate is inserted in the candidates collection. Table \ref{table:case-study:impl:candidates:schema} shows the schema definition of a document of the candidates collection.

\begin{table}[H]
    \centering
        \begin{tabular}{| l | l |} 
            \hline
                \texttt{image\_key} & An integer which identifies an image \\
            \hline
                \texttt{usno\_band} & The \usno band of the image \\
            \hline
                \texttt{panstarr\_band} & The \panstarrs band of the image \\
            \hline
                \texttt{v} & \multicolumn{1}{m{8cm}|}{The \texttt{v} value of this candidate computed by the AI--Crawler} \\
            \hline
        \end{tabular}
    \caption{A description of the schema of a document of the candidates collection.}
    \label{table:case-study:impl:candidates:schema}
\end{table}
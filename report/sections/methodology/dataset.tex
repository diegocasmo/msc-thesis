\section{Datasets} \label{sect:meth:datasets}

% What is USNO-B1.0?
% What is Pan-STARRS1?
The \mlblink algorithm was designed with the goal of aiding astronomers in the \vasco initiative to find interesting observations for further analysis. A subset of the \usno and \panstarrs datasets was used to implement and test the algorithm. \newline

\usno is an all-sky catalog composed from multiple sky surveys during the interval from 1949 to 2002 \cite{web:caltech:usno} that indicates positions, proper motions, star/galaxy estimators and other astronomical features for 1,042,618,261 objects derived from 3,643,201,733 distinct observations \cite{web:ap-i:usno}. Pan--STARRS is a system for wide-field astronomical imaging developed and operated by the Institute for Astronomy at the University of Hawaii. \panstarrs is the first part of Pan--STARRS to be completed and is the basis for both Data Releases 1 and 2 (DR1 and DR2). \panstarrs DR1 was released on December 19, 2016 \cite{web:stsci:panstarrs}.  \newline

The subset consisted of a total of 1000 unique cases in each dataset, each described across different bands. The \usno subset used a total of five bands (\texttt{blue1}, \texttt{blue2}, \texttt{red1}, \texttt{red2}, and \texttt{ir}), while \panstarrs subset used a total of three bands (\texttt{g}, \texttt{r}, and \texttt{z}). Table \ref{table:case-study:intro:datasets-mapping} shows how each of these bands are related to one another in \usno and \panstarrs respectively. Lastly, the subsets' images were all in gray--scale format for all dataset bands. 

\begin{table}[H]
    \centering
        \begin{tabular}{| c | c |} 
            \hline
                \usno Band & \panstarrs Band \\
            \hline
                \texttt{blue1} & \texttt{g} \\
            \hline
                \texttt{blue2} & \texttt{g} \\
            \hline
                \texttt{red1} & \texttt{r} \\
            \hline
                \texttt{red2} & \texttt{r} \\
            \hline
                \texttt{ir} & \texttt{z} \\
            \hline
        \end{tabular}
    \caption{Mappings which specify how each band in \usno is related to a band in \panstarrs or vice--versa.}
    \label{table:case-study:intro:datasets-mapping}
\end{table}

\subsection{Explicit Representation} \label{subsect:meth:explicit}

The explicit representation of the learned weights simply stores these weights in a matrix $\vect{D}$. The matrix $\vect{D}$ is then used to compute the value $v$ of a mission, as well as updating it to learn new information.

Algorithm \ref{pscode:compute-v:explicit} shows pseudo--code which given a mission setup will return the $v$ value of such mission. It is important to note that the very first time the matrix $\vect{D}$ is retrieved, all of its weights are equal to $0$ (i.e. it has not learned anything yet), and therefore any mission given to it will return a $v$ value equal to $0$.

\vspace{0.4cm}
\begin{algorithm}[H]
    \SetAlgoLined
    \SetKwProg{FMain}{compute\_v\_explicit}{:}{end}
        \FMain{$i$, $j$} {
            Let $i$ be an image key, and $j$ be an image band \\
            Retrieve image $\vect{x}$ according to $i$, $j$ from \usno \\
            Retrieve image $\vect{y}$ according to $i$, $j$ from \panstarrs \\
            Retrieve weights matrix $\vect{D}$ \\
            Compute $v = \vect{x}^T \vect{D} \vect{y}$ \\
            \Return $v$
        }
    \caption{Pseudo--code for computing the value $v$ for a mission setup using the explicit definition of the matrix $\vect{D}$.}
    \label{pscode:compute-v:explicit}
\end{algorithm}
\vspace{0.4cm}

Algorithm \ref{pscode:update-d:explicit} shows how the weights of the matrix $\vect{D}$ are updated in order to learn new information.

\vspace{0.4cm}
\begin{algorithm}[H]
    \SetAlgoLined
    \SetKwProg{FMain}{update\_d\_explicit}{:}{end}
        \FMain{$i$, $j$} {
            Let $i$ be an image key, and $j$ be an image band \\
            Retrieve image $\vect{x}$ according to $i$, $j$ from \usno \\
            Retrieve image $\vect{y}$ according to $i$, $j$ from \panstarrs \\
            $\vect{D} \gets \vect{D} + \vect{x}\vect{y}^T$ \\
        }
    \caption{Pseudo--code for updating the explicit representation of the matrix $\vect{D}$.}
    \label{pscode:update-d:explicit}
\end{algorithm}
\vspace{0.4cm}

The weights learned by the \mlblink algorithm must be persisted in order for multiple crawlers to be able to read and write from the matrix $\vect{D}$ at different time steps. Even though the explicit representation using the matrix $\vect{D}$ provides a simple way to describe what the algorithm has learned and how it can be used to learn new information, storing, retrieving, and updating such weights might represent a performance issue depending on the size of the vectors $\vect{x}$ and $\vect{y}$.
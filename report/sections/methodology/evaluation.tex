\section{Evaluation} \label{sect:meth:evaluation}

As mentioned earlier, the \mlblink algorithm is a recommender system. That being said, it can also be described as a binary classifier, since for any observation it attempts to ``assign'' a label to it as either an anomaly or a non--anomaly. As a result, the receiver operating characteristic (ROC) curve is a suitable technique to study and understand the capability of a model as its discrimination threshold is changed. \newline

Before diving into the ROC curve, let us first define a confusion matrix in terms of the \mlblink algorithm. A confusion matrix is simply a table which allows to visualize the performance of a model by showing what label the model assigns to the observations in comparison to the real labels of these observations. A confusion matrix for the \mlblink algorithm is shown in table \ref{table:confusion-matrix}.

\begin{table}[H]
    \centering
    \begin{tabular}{cc|cc}
        \multicolumn{2}{c}{}
        & \multicolumn{2}{c}{Actual} \\
        &       &   Anomaly &   Non--anomaly \\ 
        \cline{2-4}
        \multirow{2}{*}{Predicted}
            & Anomaly         & TP   & FP  \\
            & Non--anomaly    & FN   & TN  \\ 
            \cline{2-4}
    \end{tabular}
    \caption{A confusion matrix table for the \mlblink algorithm. The model predictions are categorized as true positive (TP), false positive (FP), false negative (FP), and true negative (TN).}
    \label{table:confusion-matrix}
\end{table}

The ROC curve is defined as the a plot of the true positive rate (TPR) against the false positive rate (FPR). The TPR is defined as in equation \ref{eq:tpr}, where TP corresponds to the true positives (top--left quadrant of the confusion matrix in table \ref{table:confusion-matrix}), and P (positives) is the total number of real anomalies in the dataset. The FPR formula is shown in equation \ref{eq:fpr}, where FP represents the false positive (top--right quadrant in table \ref{table:confusion-matrix}), and N (negatives) is the real number of non--anomalies in the dataset.

\begin{equation} \label{eq:tpr}
    \text{TPR} = \dfrac{\text{TP}}{\text{P}}
\end{equation}

\begin{equation} \label{eq:fpr}
    \text{FPR} = \dfrac{\text{FP}}{\text{N}}
\end{equation}

Finally, the ROC curve plot requires a model discrimination threshold to be changed, in the \mlblink algorithm the threshold is going to be the range of $v$ values computed for the missions that were selected to be crawled. The ROC curve will plot the TPR against the FPR ``sweeping'' the entire range of $v$ values in the selected missions from $\text{min}(v)$ up to $\text{max}(v)$, and for each $v$ value computing the TPR and FPR.

\todo[inline]{TODO: How to interpret the ROC curve?}
\todo[inline]{TODO: What is the AUC and how to interpret it?}
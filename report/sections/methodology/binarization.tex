\section{Image Retrieval} \label{sect:meth:image-retrieval}

The \mlblink algorithm retrieves images from the $\beta$--pack dataset and performs a few operations in order to pre--process the images to later on evaluate them. In addition to the aforementioned dimensionality reduction through projections (or average pooling) and normalization using the L2--norm, the images are also binarized. \newline

Binarization is a process in which an input signal is transformed such that the resulting output consists of only two values. The \mlblink algorithm uses binarization with a fixed threshold (one for each source) as a pre--processing technique when retrieving missions. Algorithm \ref{pscode:retrieve-vector} shows pseudo--code which describes how a mission's vector is retrieved using binarization, dimensionality reduction (as described in section \ref{sect:theory:dim-reduction}), and normalization (section \ref{sect:theory:norm}). This process is applicable to both $\vect{x}$ and $\vect{y}$, and it is described in terms of $\vect{z}$ for illustrative purposes only. Note line number $5$ is replaced by average pooling as a dimensionality reduction technique when appropriate.

\vspace{0.4cm}
\begin{algorithm}[H]
    \SetAlgoLined
    \SetKwProg{FMain}{retrieve\_vector}{:}{end}
        \FMain{$i$, $j$, $n_{\text{projections}}$} {
            Let $i$ be an image key, and $j$ be an image band \\
            Retrieve image $\vect{z}_{i,j}$ according to $i$, $j$ from $\vect{z}$ source \\
            $\vect{bw} \gets \vect{z}_{i,j}$ binarized with fixed threshold $t_z$ \\
            $\vect{zs} \gets \vect{bw} \cdot \vect{P}$ where $\vect{P}$ size is $n_{projections} \times n_z$ (or use average pooling) \\
            \Return $\vect{zs}$ normalized using the L2--norm
        }
    \caption{Pseudo--code to retrieve a vector given an image key $i$, an image band $j$, and the desired number of projections to use for dimensionality reduction.}
    \label{pscode:retrieve-vector}
\end{algorithm}
\vspace{0.4cm}
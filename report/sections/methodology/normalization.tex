\section{Normalization} \label{sect:meth:norm}

% What is normalization?
Normalization refers to the process of accommodating the values of observations so that their unit of measurement does not affect their contribution when compared to one another. Normalization essentially drops the unit of measurement from the observations, and as a result, it allows to examine observations that come from distinct places in a notionally common scale.  \newline

% Why use normalization?
As pointed out in section \ref{sect:meth:intro}, a mission is compromised of two images. Since these two images might have been acquired using different devices and/or software processing techniques, normalization is required in order to use a common ``scale'' between these two observations.  \newline

% How does the ML-Blink algorithm use normalization?
The \mlblink algorithm uses the L2--norm as defined in equation \ref{eq:l2-norm} to normalize the images vectors. The normalization is performed by dividing each component of a vector by the vector's L2--norm. The resulting vector has the characteristic that the sum of each of the vector's component squared sums up to 1.

\begin{equation} \label{eq:l2-norm}
    \lVert \vect{x} \rVert_{2} = \sqrt{x^2_1 + x^2_2 + ... + x^2_n}
\end{equation}
\subsection{Introduction} \label{subsect:meth:intro}

An arrangement of two images from the same location of the night sky from distinct datasets in their corresponding bands is defined as a mission. The goal of the \mlblink algorithm is then to ``crawl'' these missions in order to recommend those that are more likely to contain an anomaly. In order to do so, the \mlblink algorithm will learn what non--anomalies look like, randomly select a set of missions to process, and recommend those that are most different from the non--anomalies it has learned. The recommended mission is referred to as a candidate.

Algorithm \ref{pscode:ml-blink:explicit} shows the basic building block of what the \mlblink algorithm does, where the time steps represent when the algorithm is called to crawl for a new candidate. The value $v$ of a mission defines how similar it is to what the \mlblink algorithm has learned. Since the \mlblink algorithm is designed to learn what non--anomalies are, retrieving the mission with the minimum value $v$ of all that were crawled represents the one that is most dissimilar to what the \mlblink algorithm knows at that particular time step.

\vspace{0.4cm}
\begin{algorithm}[H]
    \SetAlgoLined
    \SetKwProg{FMain}{ml\_blink}{:}{end}
        \FMain{} {
            \For{$t = 0, 1, 2, ...$} {
                Randomly select a set of missions to crawl \\
                For each mission, compute its $v$ value \\
                Select mission with $\text{min}(v)$ as a candidate \\
            }
        }
    \caption{Pseudo--code for the basic building block of the \mlblink algorithm.}
    \label{pscode:ml-blink:explicit}
\end{algorithm}
\vspace{0.4cm}

After a set of missions has been selected, computing their corresponding $v$ value is what will differ depending on how the weights learned by the algorithm are represented.
\section{Crawling Candidates} \label{sect:meth:intro}

Algorithm \ref{pscode:ml-blink} shows the basic building block of what the \mlblink algorithm does, where the time steps represent when the algorithm is called to crawl for a new candidate. The value $v$ of a mission defines how similar it is to what the \mlblink algorithm has learned. Since the \mlblink algorithm is designed to learn what non--anomalies are, retrieving the mission with the minimum value $v$ of all that were crawled represents the one that is most dissimilar to what the \mlblink algorithm knows at that particular time step.

\vspace{0.4cm}
\begin{algorithm}[H]
    \SetAlgoLined
    \SetKwProg{FMain}{crawl\_candidates}{:}{end}
        \FMain{} {
            \For{$t = 0, 1, 2, ...$} {
                Select a set of missions to crawl \\
                For each mission, compute its $v$ value \\
                Select mission with $\text{min}(v)$ as a candidate \\
            }
        }
    \caption{Pseudo--code for the basic building block of the \mlblink algorithm.}
    \label{pscode:ml-blink}
\end{algorithm}
\vspace{0.4cm}

After a set of missions has been selected, computing their corresponding $v$ value is what will differ depending on how the weights in the matrix $\vect{D}$ learned by the algorithm are represented. It is also important to note that the very first time the matrix $\vect{D}$ is retrieved, all of its weights are equal to $0$ (i.e. it has not learned anything yet), and therefore any mission given to it will return a $v$ value equal to $0$. If multiple missions are tied for the minimum value $v$, the \mlblink algorithm will randomly select a mission within those in the tie as the candidate.
\section{Background and Motivation} \label{sect:intro:background}

% What is VASCO?
% Why is such problem important to study?
The ``Vanishing and Appearing Sources during a Century of Observations'' (\vasco) \cite{web:vasco} initiative aims at finding inexplicable effects among all-sky surveys. The \vasco project is a collaboration between astronomers and information technology researchers, and incorporates explicitly a component of citizen science. The study of differences among all-sky surveys could lead to interesting scientific findings, like new astrophysical phenomena or interesting targets for follow-up by the Search for Extraterrestrial Intelligence (\seti) observations. \newline

% What is the problem?
Previous work done in \cite{article:our-sky}, mostly based on manual comparisons, identified a vanishing point source by comparing the \usno sky survey catalog with the Sloan Digital Sky Survey (SDSS). The study of the night sky from multiple surveys to examine time variations is also described in \cite{article:two-epoch-catalog}, where a catalog with a total of 43,647,887 observations from USNO-B and SDSS was created and the issues encountered while doing so discussed. In both studies it is clear the enormous scale of existing sky surveys motivates the development of efficient computational tools, with an exciting role given to machine learning (\ml) due to its capacity to deal with data-intensive processes. \newline

% What is the goal?
The precise objective of this project is to implement and test an \ml algorithm which uses a data-driven approach to attempt to learn what features characterize interesting candidates from the historical sky survey observations. The \ml component is described as \mlblink and it is based on methods of active and online semi-supervised learning. \mlblink is named after the blink comparator; a 19th century viewing device invented by physicist Carl Pulfrich used by astronomers to discover differences between two images of the night sky. \newline

% How does the ML-Blink algorithm fit within VASCO?
Within the \vasco initiative, the \mlblink algorithm will be used in order to identify anomalies that might be present in the historical sky survey observations. These surveys contain images from the same location in the night sky, but from distinct times. An arrangement of two images from the same location of the night sky from distinct datasets is defined as a mission. The goal of the \mlblink algorithm is then to ``crawl'' these missions in order to recommend those that are more likely to contain an anomaly (i.e. a recommender system). In order to do so, the \mlblink algorithm will learn what non--anomalies look like,  select a set of missions to process, and recommend those that are most different from the non--anomalies it has learned. The recommended mission is referred to as a candidate.
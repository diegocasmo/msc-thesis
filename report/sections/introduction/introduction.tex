\section{Introduction} \label{sect:intro}

\subsection{Background and Motivation} \label{subsect:intro:background}

\subsection{Recommender Systems} \label{subsect:intro:recommender-systems}

% What is a recommender system?
A recommender system is a computer software which allows to provide product suggestions that serve a certain purpose to an entity. The entity to which such recommendation is provided is usually referred to as the user, while the product being recommended is commonly referred to as an item \cite{book:rs}.

% Why recommender systems?
% What is the goal of a recommender system?
The usage of a recommender system is typically motivated by the existence of a set of predefined objectives to optimize and a possibly overwhelming number of items to choose from. A recommender system's goal is to maximize the established set of objectives, and it can be accomplished by the use of a data--driven approach which attempts to learn existing dependencies among users and items.

% How does a simple recommender system work?
As an example, consider an online bookstore that uses a recommender system to suggest books to its users. Such system might utilize explicit feedback such as a star rating system (e.g., 0--5), or implicit feedback like browsing for a title or buying a book to infer its users interests. The recommender system prediction based off the data aforementioned can then be used to increase profit and user engagement in the platform.

\subsection{Outline of Thesis} \label{subsect:intro:outline}

\chapter{Conclusion and Future Work} \label{ch:conclusion}

% Introduction
After a general introduction to recommender systems, supervised, online, and active learning, the \mlblink algorithm was introduced and some of its mathematical properties were intuitively explained. The relation of the \mlblink algorithm to the aforementioned learning techniques was described, and finally, the concepts of normalization and dimensionality reduction were defined and illustrated in terms of how these are used in the \mlblink algorithm. \newline

% Methodology
The methodology section outlined how the theoretical definition of the \mlblink algorithm was approached for this particular work. A short explanation of the dataset the \mlblink algorithm would be evaluated on was given as well. Next, the implicit representation of the weight matrix $\vect{D}$, how each observation from the dataset is retrieved, and the parallelization of the \mlblink algorithm were studied. Lastly, an evaluation technique based on the ROC curve, AUC, and the number of anomalies found was presented. \newline

% Case study (ML-Blink UI and ML-Blink API)
The case study section described in detail a UI that allows users to match two images of the same location in the night sky from distinct datasets, and an API that processes and persists data created by the UI. The architecture constructed for the UI and API to interact with each other along with the required technologies to handle asynchronous tasks execution and data persistence were motivated and explained. \newline

% Results
The results section focused on evaluating the \mlblink algorithm in isolation, that is, without the \mlblinkui. The \mlblink algorithm using projections for dimensionality reduction was able to achieve an AUC around the $0.70$ range when crawling all \panstarrs color--bands, and find $2$--$4$ artificial anomalies out of $7$. Since the artificial anomalies the \mlblink algorithm found were always in the \panstarrs color--band \texttt{g}, in order to better understand the algorithm's performance, each of the \panstarrs color--bands was evaluated in isolation. The \mlblink algorithm achieved an excellent AUC, typically above $0.90$, when crawling the \panstarrs color--band \texttt{g} in isolation. Nonetheless, the \mlblink algorithm performance when crawling the \panstarrs color--bands \texttt{r} or \texttt{z} was poor, consistently achieving an AUC below $0.5$ and no artificial anomalies were found at all. Finally, a few configurations of average pooling as a dimensionality reduction technique were tested, but ultimately no further experiments were conducted with this technique due to the instability and lack of reproducibility encountered in these evaluations. \newline

% Future Work
Before introducing the binarization step when retrieving an observation from the dataset, the \mlblink algorithm was completely unable to find any of the anomalies manually inserted in the $\beta$--pack observations. This suggests more work can be done in the pre--processing steps. For instance, the binarization threshold is fixed, and furthermore, it has the same value regardless of what \panstarrs color--band is being crawled. It might be worth experimenting with automatic thresholding, and even consider using different pre--processing steps for each color--band. \newline

The current implementation of \mlblink crawls the same $5005$ observations from the $\beta$--pack in all time steps. Eventually, when using a larger dataset, it must be decided how to crawl each observation. The larger datasets could use a complete crawler from start to finish (e.g. 0--999, 1000--1999, 2000--2999, etc), or a specified number of observations might be randomly selected in each time step within predefined lower/upper bound limits. \newline

Within the proposed \vasco software architecture, this work focused the least in the \mlblinkui. Nonetheless, the \mlblinkui is essential for the rest of the project, as it provides the UI that allows users to label observations and ultimately feed the \mlblink algorithm. Possible improvements to the \mlblinkui include gamification strategies to incentive users to solve more missions with a high degree of accuracy, the addition of a rotation feature which allows to match observations that could not be matched otherwise, and experimentation with other matching algorithms to determine whether a more intuitive and robust matching algorithm could be designed. \newline

\todo[inline]{TODO: What else for future work?}
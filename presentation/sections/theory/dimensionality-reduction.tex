\begin{frame}{Dimensionality Reduction}
    \begin{itemize}
        \item Dimensionality refers to the number of features that make up an observation
        \item The curse of dimensionality is the manifestation of all phenomena that occurs when dealing with high--dimensional data \cite{article:curse-of-dim}
        \item Dimensionality reduction is the process of reducing the number of features that describe an observation
            \begin{itemize}
                \item Feature selection: selecting a subset of the original features
                \item Feature extraction: deriving new features from the original features
            \end{itemize}
    \end{itemize}
\end{frame}

\begin{frame}{Dimensionality Reduction}
    \begin{itemize}
        \item Projections: use a linear inner product to project a pair of items $(\vect{x}_i, \vect{y}_j)$ to a lower dimension
        \begin{equation} \label{eq:proj}
            \begin{split}
                \vect{p} 
                &= 
                    \vect{x}_{i} \cdot \vect{P} \\
                &=
                    \vect{x}_{i} \cdot
                    \begin{bmatrix}
                        1 & 1 & 1 & 0 & 0 & 0 & 0 & 0 & 0 \\
                        0 & 0 & 0 & 1 & 1 & 1 & 0 & 0 & 0 \\
                        0 & 0 & 0 & 0 & 0 & 0 & 1 & 1 & 1 \\
                    \end{bmatrix} \\  
                &=
                    \begin{bmatrix}
                        x_{i,1} + x_{i,2} + x_{i,3} & x_{i,4} + x_{i,5} + x_{i,6} & x_{i,7} + x_{i,8} + x_{i,9}
                    \end{bmatrix} \\
            \end{split}
        \end{equation}
    \end{itemize}
\end{frame}

% \begin{frame}{Dimensionality Reduction}
%     \begin{itemize}
%         \item Pooling: change a collective feature representation into a new, more usable one that maintains important information while eliminating irrelevant detail \cite{article:pooling}
%             \begin{figure}[H]
%                 \centering
%                 \begin{tikzpicture}
%                     \fill [c1] (0, 0) rectangle (2, 2);
%                     \fill [c2]   (2, 0) rectangle (4, 2);
%                     \fill [c4] (0, 2) rectangle (2, 4);
%                     \fill [c5]   (2, 2) rectangle (4, 4);
                    
%                     \fill [c1]  (4.9, 1) rectangle (5.9, 2);
%                     \fill [c2]   (5.9, 1) rectangle (6.9, 2);
%                     \fill [c4]  (4.9, 2) rectangle (5.9, 3);
%                     \fill [c5]   (5.9, 2) rectangle (6.9, 3);
                
%                     \foreach \i in {\xMin,...,\xMax} {
%                         \draw [very thin,gray] (\i,\yMin) -- (\i,\yMax)  node [below] at (\i,\yMin) {};
%                     }
%                     \foreach \i in {\yMin,...,\yMax} {
%                         \draw [very thin,gray] (\xMin,\i) -- (\xMax,\i) node [left] at (\xMin,\i) {};
%                     }
                
%                     \foreach \i in {\xMin,2,...,\xMax} {
%                         \draw [thick,gray] (\i,\yMin) -- (\i,\yMax)  node [below] at (\i,\yMin) {};
%                     }
%                     \foreach \i in {\yMin,2,...,\yMax} {
%                         \draw [thick,gray] (\xMin,\i) -- (\xMax,\i) node [left] at (\xMin,\i) {};
%                     }
            
%                     \node at (0.5, 3.5) {0};
%                     \node at (1.5, 3.5) {1};
%                     \node at (2.5, 3.5) {2};
%                     \node at (3.5, 3.5) {3};
%                     %
%                     \node at (0.5, 2.5) {4};
%                     \node at (1.5, 2.5) {5};
%                     \node at (2.5, 2.5) {6};
%                     \node at (3.5, 2.5) {7};
%                     %
%                     \node at (0.5, 1.5) {8};
%                     \node at (1.5, 1.5) {9};
%                     \node at (2.5, 1.5) {10};
%                     \node at (3.5, 1.5) {11};
%                     %
%                     \node at (0.5, 0.5) {12};
%                     \node at (1.5, 0.5) {13};
%                     \node at (2.5, 0.5) {14};
%                     \node at (3.5, 0.5) {15};
                
%                     \foreach \i in {\xMinR,...,\xMaxR} {
%                         \draw [thick,gray] (\i,\yMinR) -- (\i,\yMaxR)  node [below] at (\i,\yMinR) {};
%                     }
%                     \foreach \i in {\yMinR,...,\yMaxR} {
%                         \draw [thick,gray] (\xMinR,\i) -- (\xMaxR,\i) node [left] at (\xMinR,\i) {};
%                     }
                
%                     \node at (5.4, 2.5) {2.5};
%                     \node at (6.4, 2.5) {4.5};
%                     %
%                     \node at (5.4, 1.5) {10.5};
%                     \node at (6.4, 1.5) {12.5};
                
%                     \draw [decorate,decoration={brace,amplitude=4pt},xshift=-2pt,yshift=0pt]
%                         (0,2) -- (0,4) node [black,midway,xshift=-0.3cm] {\footnotesize $2$};
                
%                     \draw [decorate,decoration={brace,amplitude=4pt},xshift=0pt,yshift=2pt]
%                         (0,4) -- (2,4) node [black,midway,yshift=+0.3cm] {\footnotesize $2$};
%                 \end{tikzpicture}
%                 \label{fig:average-pooling}
%             \end{figure}
%     \end{itemize}
% \end{frame}

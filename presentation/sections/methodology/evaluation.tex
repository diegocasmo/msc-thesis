\begin{frame}{Evaluation}
    \begin{itemize}
        \item Receiver operating characteristic (ROC) curve is a suitable technique to study an \mlblink model as its discrimination threshold is changed
        \item A confusion matrix
            \begin{table}[H]
                \begin{tabular}{cc|cc}
                    \multicolumn{2}{c}{}
                    & \multicolumn{2}{c}{Actual} \\
                    &       &   Anomaly &   Normal \\ 
                    \cline{2-4}
                    \multirow{2}{*}{Predicted}
                        & Anomaly         & TP   & FP  \\
                        & Normal    & FN   & TN  \\ 
                        \cline{2-4}
                \end{tabular}
                \caption{A confusion matrix table for the \mlblink algorithm. The model predictions are categorized as true positive (TP), false positive (FP), false negative (FP), and true negative (TN).}
                \label{table:confusion-matrix}
            \end{table}
    \end{itemize}
\end{frame}

\begin{frame}{Evaluation}
    \begin{itemize}
        \item The ROC curve is a plot of the true positive rate (TPR) against the false positive rate (FPR)
        \item[\xspace]
            \begin{equation} \label{eq:tpr}
                \text{TPR} = \dfrac{\text{TP}}{\text{P}}
            \end{equation}
            
            \begin{equation} \label{eq:fpr}
                \text{FPR} = \dfrac{\text{FP}}{\text{N}}
            \end{equation}  
        \item P = positives (anomaly), N = negatives (normal)
    \end{itemize}
\end{frame}

\begin{frame}{Evaluation}
    \begin{itemize}
        \item The $v$ value will act as the model discrimination threshold
        \item The ROC curve will plot the TPR/FPR ``sweeping'' the range of $v$ values from $\text{min}(v)$ to $\text{max}(v)$
        \item Common to show the Area Under Curve (AUC) when plotting the ROC curve
        \item Equivalent to the probability that a classifier will rank a randomly chosen positive observation higher than a randomly chosen negative observation \cite{article:roc-analysis}
            \begin{itemize}
                \item Assuming normalize units are being used
            \end{itemize}
    \end{itemize}
\end{frame}
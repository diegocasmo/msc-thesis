\begin{frame}{Datasets}
    \begin{itemize}
        \item \usno is an all-sky catalog composed from multiple sky surveys during the interval from 1949 to 2002 \cite{web:caltech:usno}
        \item Pan--STARRS is a system for wide-field astronomical imaging
        \item \panstarrs is the first part of Pan--STARRS to be completed
    \end{itemize}
\end{frame}

\begin{frame}{Datasets}
    \begin{itemize}
        \item A subset of the \usno and \panstarrs datasets gathered by Johan Soodla ($\beta$--pack) was used to implement and test the algorithm
        \item The center of the image must contain a star, galaxy or artifact for at least $95$\% of the cases
        \item The subset consists of a total of 1001 unique cases in each dataset, each described across different color--bands
        \item Images are all in gray--scale format for all datasets color--bands
    \end{itemize}
\end{frame}

\begin{frame}{Datasets}
    \begin{itemize}
        \item Color--bands mappings
            \begin{table}[H]
                \centering
                    \begin{tabular}{| c | c |} 
                        \hline
                            \usno Band & \panstarrs Band \\
                        \hline
                            \texttt{blue1} & \texttt{g} \\
                        \hline
                            \texttt{blue2} & \texttt{g} \\
                        \hline
                            \texttt{red1} & \texttt{r} \\
                        \hline
                            \texttt{red2} & \texttt{r} \\
                        \hline
                            \texttt{ir} & \texttt{z} \\
                        \hline
                    \end{tabular}
                \caption{Mappings which specify how each color--band in \usno is related to a color--band in \panstarrs or vice--versa.}
                \label{table:case-study:intro:datasets-mapping}
            \end{table}
    \end{itemize}
\end{frame}
\begin{frame}{Results}
    \begin{itemize}
        \item The \mlblink algorithm is evaluated in isolation (no \mlblinkui)
        \item Compare the \mlblink algorithm to randomly searching for anomalies
            \begin{itemize}
                \item Recommend an item with probability of $1/5005 = 0.0002$
                \item In 200 time steps $7 \times 1/5005 \times 200 = 0.27$ anomalies
            \end{itemize}
        \item A total of 7 artificial anomalies were created
    \end{itemize}
\end{frame}

\begin{frame}{Results}
    \begin{table}[H]
        \centering
            \begin{tabular}{| c | c | c |}
                \hline
                  Image Key & \usno Band & \panstarrs Band \\
                \hline
                  13 & \texttt{blue1} & \texttt{g} \\
                \hline
                  13 & \texttt{blue2} & \texttt{g} \\
                \hline
                  56 & \texttt{blue1} & \texttt{g} \\
                \hline
                  56 & \texttt{blue2} & \texttt{g} \\
                \hline
                  679 & \texttt{ir} & \texttt{z} \\
                \hline
                  831 & \texttt{red1} & \texttt{r} \\
                \hline
                  831 & \texttt{red2} & \texttt{r} \\
                \hline
            \end{tabular}
        \caption{Complete list of anomalies that were created to evaluate the \mlblink algorithm.}
    \end{table}
\end{frame}

\begin{frame}{Results}
    \begin{figure}
      \centering
      \includegraphics[
            width=0.7\textwidth,
            keepaspectratio
      ]{report/images/results/250_all_bands/250_all_bands_3_anomalies_t_200.png}
      \caption{ROC curve and AUC of the evaluation of the \mlblink algorithm using 250 projections for a total of 200 time steps.}
    \end{figure}
\end{frame}

\begin{frame}{Results}
    \begin{figure}
      \centering
      \includegraphics[
            width=0.7\textwidth,
            keepaspectratio
      ]{report/images/results/250_all_bands/found_250_all_bands_3_anomalies_t_200.png}
      \caption{Evaluation of known anomalies (that the \mlblink algorithm found) versus normal observations in comparison to the $\min(v)$ value at each time step.}
    \end{figure}
\end{frame}

\begin{frame}{Results}
    \begin{figure}
      \centering
      \includegraphics[
            width=0.7\textwidth,
            keepaspectratio
      ]{report/images/results/250_all_bands/not_found_250_all_bands_4_anomalies_t_200.png}
      \caption{Evaluation of known anomalies (that the \mlblink algorithm could not find) versus normal observations in comparison to the $\min(v)$ value at each time step.}
      \label{fig:evaluation:v-versus-t:not-found}
    \end{figure}
\end{frame}

\begin{frame}{Results}
    \begin{itemize}
        \item All artificial anomalies evaluated close to the $\min(v)$ of any time step are in the \panstarrs color--band \texttt{g}
        \item Next step is to evaluate each \panstarrs band in isolation
    \end{itemize}
\end{frame}

\begin{frame}{Results}
    \begin{table}[H]
        \centering
            \begin{tabular}{| c | c | c |}
                \hline
                  Image Key & \usno Band & \panstarrs Band \\
                \hline
                  13 & \texttt{blue1} & \texttt{g} \\
                \hline
                  13 & \texttt{blue2} & \texttt{g} \\
                \hline
                  56 & \texttt{blue1} & \texttt{g} \\
                \hline
                  56 & \texttt{blue2} & \texttt{g} \\
                \hline
                  679 & \texttt{ir} & \texttt{z} \\
                \hline
                  831 & \texttt{red1} & \texttt{r} \\
                \hline
                  831 & \texttt{red2} & \texttt{r} \\
                \hline
            \end{tabular}
        \caption{Complete list of anomalies that were created to evaluate the \mlblink algorithm.}
    \end{table}
\end{frame}

\begin{frame}{Results}
    \begin{figure}[H]
        \centering
        \includegraphics[
            width=0.7\textwidth,
            keepaspectratio
        ]{report/images/results/250_panstarr_g/250_panstarr_g_3_anomalies_t_200.png}
        \caption{ROC curve and AUC achieved by the \mlblink algorithm when evaluating observations in the \panstarrs color--band \texttt{g} only (versus \usno color--bands \texttt{blue1} and \texttt{blue2}) using 250 projections for dimensionality reduction.}
    \end{figure}
\end{frame}


\begin{frame}{Results}
    \begin{figure}[H]
        \centering
        \includegraphics[
            width=0.7\textwidth,
            keepaspectratio
        ]{report/images/results/250_panstarr_g/found_250_panstarr_g_3_anomalies_t_200.png}
        \caption{Evaluation of known anomalies that were found by the \mlblink algorithm when evaluating the \panstarrs color--band \texttt{g} (versus \usno color--bands \texttt{blue1} and \texttt{blue2}) in isolation using 250 projections for dimensionality reduction.}
    \end{figure}
\end{frame}

\begin{frame}{Results}
    \begin{figure}[H]
        \centering
        \includegraphics[
            width=0.7\textwidth,
            keepaspectratio
        ]{report/images/results/250_panstarr_r/250_panstarr_r_0_anomalies_t_200.png}
        \caption{ROC curve and AUC achieved by the \mlblink algorithm when evaluating observations in the \panstarrs color--band \texttt{r} only (versus \usno color--bands \texttt{red1} and \texttt{red2}) using 250 projections for dimensionality reduction.}
    \end{figure}
\end{frame}

\begin{frame}{Results}
    \begin{figure}[H]
        \centering
        \includegraphics[
            width=0.7\textwidth,
            keepaspectratio
        ]{report/images/results/250_panstarr_r/not_found_250_panstarr_r_2_anomalies_t_200.png}
        \caption{Evaluation of known anomalies that were not found by the \mlblink algorithm when evaluating the \panstarrs color--band \texttt{r} (versus \usno color--bands \texttt{red1} and \texttt{red2}) in isolation using 250 projections for dimensionality reduction.}
    \end{figure}
\end{frame}

\begin{frame}{Results}
    \begin{figure}[H]
        \centering
        \includegraphics[
            width=0.7\textwidth,
            keepaspectratio
        ]{report/images/results/250_panstarr_z/250_panstarr_z_0_anomalies_t_200.png}
        \caption{ROC curve and AUC achieved by the \mlblink algorithm when evaluating observations in the \panstarrs color--band \texttt{z} only (versus \usno color--band \texttt{ir}) using 250 projections for dimensionality reduction.}
        \label{fig:evaluation:roc:panstarrs:z}
    \end{figure}
\end{frame}

\begin{frame}{Results}
    \begin{figure}[H]
        \centering
        \includegraphics[
            width=0.7\textwidth,
            keepaspectratio
        ]{report/images/results/250_panstarr_z/not_found_250_panstarr_z_1_anomalies_t_200.png}
        \caption{Evaluation of known anomalies that were not found by the \mlblink algorithm when evaluating the \panstarrs color--band \texttt{z} (versus \usno color--bands \texttt{ir}) in isolation using 250 projections for dimensionality reduction.}
    \end{figure}
\end{frame}

\begin{frame}{Results}
    \begin{itemize}
        \item Same experiments were conducted using average pooling with a kernel of size $7 \times 7$ for dimensionality reduction
        \item Results when using average pooling were unstable and difficult to reproduce
        \item It was decided to not continue these experiments further due to their instability and lack of reproducibility
    \end{itemize}
\end{frame}

\begin{frame}{Results}
    \begin{figure}[H]
        \centering
        \includegraphics[
            width=0.7\textwidth,
            keepaspectratio
        ]{report/images/results/7x7_all_bands_4/7x7_all_bands_2_anomalies_t_200.png}
        \caption{ROC curve and AUC when using average pooling with a kernel of size $7 \times 7$ for a total of 200 time steps}
    \end{figure}
\end{frame}

\begin{frame}{Results}
    \begin{figure}[H]
        \centering
        \includegraphics[
            width=0.7\textwidth,
            keepaspectratio
        ]{report/images/results/7x7_all_bands_4/found_7x7_all_bands_2_anomalies_t_200.png}
        \caption{Evaluation of the $v$ value of known anomalies and a normal observation as the \mlblink algorithm is taught over time when using average pooling for dimensionality reduction}
    \end{figure}
\end{frame}